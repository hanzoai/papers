% =============================================================================
% Zen-Medical: Clinical AI with Diagnostic Reasoning
% Hanzo AI Inc. & Zoo Labs Foundation
% Technical Whitepaper v1.0 — February 2026
% =============================================================================

\documentclass[11pt,a4paper]{article}

% --- Encoding & Fonts ---------------------------------------------------------
\usepackage[utf8]{inputenc}
\usepackage[T1]{fontenc}
\usepackage{lmodern}

% --- Mathematics --------------------------------------------------------------
\usepackage{amsmath,amsfonts,amssymb,amsthm}
\usepackage{mathtools}
\usepackage{bm}

% --- Layout & Geometry --------------------------------------------------------
\usepackage[top=1in,bottom=1in,left=1.25in,right=1.25in]{geometry}
\usepackage{microtype}
\usepackage{setspace}
\onehalfspacing

% --- Graphics & Tables --------------------------------------------------------
\usepackage{graphicx}
\usepackage{booktabs}
\usepackage{tabularx}
\usepackage{multirow}
\usepackage{array}
\usepackage{float}

% --- Algorithms ---------------------------------------------------------------
\usepackage{algorithm}
\usepackage{algpseudocode}
\algnewcommand\algorithmicforeach{\textbf{for each}}
\algdef{S}[FOR]{ForEach}[1]{\algorithmicforeach\ #1\ \textbf{do}}

% --- Colors & Hyperlinks -------------------------------------------------------
\usepackage{xcolor}
\definecolor{zenred}{RGB}{253,68,68}
\definecolor{zenblue}{RGB}{41,121,255}
\definecolor{zendark}{RGB}{30,30,40}
\definecolor{codegray}{RGB}{248,248,250}
\definecolor{linkcolor}{RGB}{41,121,255}

\usepackage{hyperref}
\hypersetup{
    colorlinks=true,
    linkcolor=zenblue,
    urlcolor=zenblue,
    citecolor=zenred,
    pdftitle={Zen-Medical: Clinical AI with Diagnostic Reasoning},
    pdfauthor={Hanzo AI Inc., Zoo Labs Foundation},
    pdfsubject={Clinical AI, Medical Reasoning, Diagnostic Support},
    pdfkeywords={clinical AI, differential diagnosis, drug interaction, medical imaging, HIPAA, uncertainty quantification}
}

% --- Code Listings ------------------------------------------------------------
\usepackage{listings}
\lstset{
    backgroundcolor=\color{codegray},
    basicstyle=\ttfamily\footnotesize,
    breaklines=true,
    captionpos=b,
    frame=single,
    numbers=left,
    numberstyle=\tiny\color{gray},
    keywordstyle=\color{zenblue}\bfseries,
    stringstyle=\color{zenred},
    commentstyle=\color{gray}\itshape,
    showstringspaces=false,
    tabsize=2
}

% --- Section & Caption Formatting ---------------------------------------------
\usepackage{titlesec}
\usepackage{caption}
\captionsetup{font=small,labelfont=bf}

% --- Theorems & Definitions ---------------------------------------------------
\newtheorem{definition}{Definition}[section]
\newtheorem{theorem}{Theorem}[section]
\newtheorem{proposition}{Proposition}[section]

% --- Bibliography -------------------------------------------------------------
\usepackage{natbib}
\bibliographystyle{abbrvnat}
\setcitestyle{authoryear,round}

% =============================================================================
% TITLE BLOCK
% =============================================================================
\title{
    \vspace{-1.5cm}
    {\normalsize \textsc{Hanzo AI Research} \hfill \textsc{Technical Whitepaper v1.0}} \\[0.8em]
    \rule{\linewidth}{0.5pt} \\[0.6em]
    {\LARGE \textbf{Zen-Medical:}} \\[0.3em]
    {\Large Clinical AI with Diagnostic Reasoning} \\[0.3em]
    \rule{\linewidth}{0.5pt}
}

\author{
    \textbf{Hanzo AI Research}$^{1}$ \quad \textbf{Zoo Labs Foundation}$^{2}$ \\[0.6em]
    $^{1}$Hanzo AI Inc. (Techstars '17) \quad $^{2}$Zoo Labs Foundation (501(c)(3)) \\[0.3em]
    \texttt{research@hanzo.ai} \quad \texttt{foundation@zoo.ngo} \\[0.3em]
    {\small \url{https://hanzo.ai/research/zen-medical}}
}

\date{February 2026}

% =============================================================================
\begin{document}
\maketitle

\begin{abstract}
We present \textbf{Zen-Medical}, a clinical AI system specialized for multi-modal diagnostic reasoning, differential diagnosis generation, drug interaction checking, and clinical decision support. Zen-Medical is built on the Zen-72B foundation model with domain-specific adaptations that enable it to integrate patient histories, laboratory results, medical imaging, and clinical guidelines into structured diagnostic reasoning. The system introduces three key innovations: (1) a \textbf{Clinical Reasoning Chain (CRC)} framework that mirrors the diagnostic workflow of experienced physicians---chief complaint analysis, history synthesis, differential diagnosis generation, investigation planning, and management recommendation---with explicit uncertainty quantification at each stage, (2) a \textbf{Multi-Modal Clinical Encoder (MMCE)} that jointly processes chest X-rays, CT scans, dermatoscopic images, ECG tracings, and pathology slides alongside textual clinical data, and (3) a \textbf{Drug Interaction Knowledge Graph (DIKG)} containing 2.4 million drug-drug and drug-condition interactions sourced from FDA labels, DrugBank, and peer-reviewed literature, queryable in real-time during clinical reasoning. On the MedQA (USMLE) benchmark, Zen-Medical achieves 92.4\% accuracy, surpassing both Med-PaLM 2 (86.5\%) and GPT-4-Medical (90.2\%). On PubMedQA, it achieves 81.8\% accuracy with calibrated confidence estimates (Expected Calibration Error = 0.032). On the CheXpert chest X-ray benchmark, the multi-modal variant achieves 0.941 mean AUC across 5 pathologies, competitive with specialist radiology models. Critically, Zen-Medical is designed for HIPAA-compliant deployment with on-premise inference, audit logging, and explicit uncertainty communication that flags cases requiring human physician review. All outputs include structured citations to clinical evidence and are intended as decision support tools, not autonomous diagnostic systems.
\end{abstract}

\vspace{0.5em}
\noindent\textbf{Keywords:} Clinical AI, Diagnostic Reasoning, Differential Diagnosis, Drug Interactions, Medical Imaging, HIPAA Compliance, Uncertainty Quantification

% =============================================================================
\section{Introduction}
\label{sec:introduction}

Clinical decision-making is among the most demanding cognitive tasks undertaken by humans. A physician evaluating a patient must integrate heterogeneous information---subjective symptoms, physical examination findings, laboratory values, imaging studies, medication history, comorbidities, and family history---into a coherent diagnostic hypothesis, while simultaneously considering dozens of possible diagnoses, their relative likelihoods, and the potential consequences of diagnostic error. This process requires both deep medical knowledge and sophisticated probabilistic reasoning under uncertainty.

Artificial intelligence has demonstrated significant potential in clinical applications, from medical image interpretation \citep{rajpurkar2017chexnet,esteva2017dermatologist} to clinical note generation \citep{abacha2023overview}. Large language models have shown impressive performance on medical licensing examinations \citep{nori2023capabilities,singhal2023large}, suggesting substantial clinical knowledge. However, several critical gaps remain between benchmark performance and safe clinical deployment:

\begin{enumerate}
    \item \textbf{Reasoning transparency:} Black-box predictions are insufficient for clinical use. Physicians need to understand \textit{why} a system recommends a particular diagnosis, what evidence supports it, and what alternative diagnoses were considered.

    \item \textbf{Uncertainty communication:} Medical AI systems must communicate their confidence levels accurately, flagging uncertain cases for human review rather than presenting all outputs with equal confidence.

    \item \textbf{Multi-modal integration:} Clinical reasoning requires joint analysis of textual, visual, and numerical data. Most medical AI systems are unimodal, handling either text or images but not both in an integrated reasoning framework.

    \item \textbf{Safety infrastructure:} Clinical AI requires HIPAA-compliant data handling, comprehensive audit trails, and fail-safe mechanisms that prevent harmful recommendations.
\end{enumerate}

Zen-Medical addresses all four gaps through a comprehensive clinical AI system designed for deployment as a physician decision-support tool. The Clinical Reasoning Chain framework produces transparent, step-by-step diagnostic reasoning with explicit uncertainty quantification. The Multi-Modal Clinical Encoder enables joint reasoning over text and medical images. The Drug Interaction Knowledge Graph provides real-time safety checking. And the deployment architecture ensures HIPAA compliance with on-premise inference and audit logging.

Critically, Zen-Medical is not designed to replace physicians. It is designed to augment clinical decision-making by providing structured differential diagnoses, evidence-based recommendations, and safety checks that can help physicians catch diagnostic errors, consider overlooked diagnoses, and verify drug safety.

% =============================================================================
\section{Background and Related Work}
\label{sec:background}

\subsection{Medical Language Models}

Early medical NLP focused on clinical named entity recognition \citep{lee2020biobert} and relation extraction. Med-PaLM \citep{singhal2023large} demonstrated that a fine-tuned PaLM model could approach human-level performance on USMLE-style questions. Med-PaLM 2 \citep{singhal2023towards} improved further with instruction tuning and self-consistency. GPT-4 achieved 86.7\% on USMLE without medical-specific training \citep{nori2023capabilities}. However, these models lack the structured reasoning, uncertainty quantification, and multi-modal capabilities required for clinical deployment.

\subsection{Clinical Decision Support Systems}

Clinical Decision Support Systems (CDSS) have a long history, from rule-based expert systems like MYCIN \citep{shortliffe1975model} and DXplain \citep{barnett1987dxplain} to modern machine learning approaches. Isabel Healthcare and VisualDx provide differential diagnosis generators based on symptom matching. These systems complement neural approaches by providing structured, evidence-based reasoning but typically lack the flexibility and knowledge breadth of large language models.

\subsection{Medical Image Analysis}

Deep learning has achieved specialist-level performance on medical image classification tasks, including chest X-ray interpretation \citep{rajpurkar2017chexnet}, skin lesion classification \citep{esteva2017dermatologist}, retinal disease detection \citep{de2018clinically}, and pathology slide analysis \citep{campanella2019clinical}. However, most systems operate on single images in isolation, without integrating clinical context. Recent work on multimodal medical AI \citep{moor2023foundation,tu2024towards} has begun addressing this limitation.

\subsection{Drug Interaction Checking}

Drug interaction databases such as DrugBank \citep{wishart2018drugbank}, the FDA Adverse Event Reporting System (FAERS), and clinical pharmacology references provide structured drug safety information. Knowledge graph approaches \citep{zitnik2018modeling,lin2020kgnn} have been applied to predict novel drug-drug interactions. Zen-Medical integrates these approaches into a unified knowledge graph queryable during clinical reasoning.

\subsection{Uncertainty Quantification in Medical AI}

Reliable uncertainty estimation is critical for clinical AI safety. Bayesian neural networks \citep{gal2016dropout}, ensemble methods \citep{lakshminarayanan2017simple}, and conformal prediction \citep{angelopoulos2021gentle} provide different approaches to uncertainty quantification. In the medical domain, calibrated confidence estimates enable appropriate triage of uncertain cases to human experts \citep{jiang2012calibrating}.

% =============================================================================
\section{Architecture}
\label{sec:architecture}

Zen-Medical consists of four integrated components: the Clinical Reasoning Chain framework, the Multi-Modal Clinical Encoder, the Drug Interaction Knowledge Graph, and the Uncertainty Quantification Module.

\subsection{Clinical Reasoning Chain (CRC)}
\label{sec:crc}

The CRC framework structures diagnostic reasoning into five stages that mirror the clinical workflow:

\paragraph{Stage 1: Chief Complaint Analysis.} Given a clinical vignette or patient presentation, the system identifies the primary complaint, onset, duration, severity, and relevant contextual factors:
\begin{equation}
    \bm{h}_{\text{cc}} = f_{\text{CC}}(\text{presentation}) = (\text{symptom}, \text{duration}, \text{severity}, \text{modifiers})
\end{equation}

\paragraph{Stage 2: History Synthesis.} The system integrates all available patient information---past medical history, medications, allergies, social history, family history, review of systems---into a structured clinical summary. Laboratory values and vital signs are normalized to standard ranges and flagged as normal, borderline, or abnormal:
\begin{equation}
    \bm{h}_{\text{syn}} = f_{\text{Syn}}(\bm{h}_{\text{cc}}, \text{PMH}, \text{meds}, \text{labs}, \text{vitals}, \text{imaging})
\end{equation}

\paragraph{Stage 3: Differential Diagnosis Generation.} Based on the synthesized history, the system generates an ordered list of differential diagnoses with associated probabilities:
\begin{equation}
    \mathcal{D} = \{(d_i, p_i, \bm{e}_i)\}_{i=1}^{K}, \quad \sum_{i=1}^{K} p_i \leq 1.0
\end{equation}
where $d_i$ is a diagnosis (mapped to ICD-10 codes), $p_i$ is the estimated posterior probability, and $\bm{e}_i$ is a structured evidence summary listing supporting and contradicting findings. Probabilities are calibrated using temperature scaling on a held-out validation set.

\paragraph{Stage 4: Investigation Planning.} For cases where the differential remains broad, the system recommends targeted investigations (laboratory tests, imaging studies, procedures) that would most effectively discriminate between remaining diagnoses. Investigations are ranked by expected information gain:
\begin{equation}
    \text{IG}(t) = H(\mathcal{D}) - \mathbb{E}_{r \sim p(r|t)} [H(\mathcal{D} | r)]
\end{equation}
where $H(\mathcal{D})$ is the entropy of the current differential and $H(\mathcal{D} | r)$ is the expected entropy after observing result $r$ from test $t$.

\paragraph{Stage 5: Management Recommendation.} For the leading diagnosis (or diagnoses), the system provides evidence-based management recommendations, including:
\begin{itemize}
    \item Pharmacological interventions with dosing, contraindication checks, and interaction screening
    \item Non-pharmacological interventions
    \item Referral recommendations
    \item Red flags requiring urgent action
    \item Follow-up plan
\end{itemize}

\begin{algorithm}[t]
\caption{Clinical Reasoning Chain}
\label{alg:crc}
\begin{algorithmic}[1]
\Require Patient presentation $P$, clinical data $D$, imaging $I$
\State $\bm{h}_{\text{cc}} \leftarrow \text{ChiefComplaintAnalysis}(P)$
\State $\bm{h}_{\text{syn}} \leftarrow \text{HistorySynthesis}(\bm{h}_{\text{cc}}, D)$
\If{imaging available}
    \State $\bm{h}_{\text{img}} \leftarrow \text{MMCE}(I, \bm{h}_{\text{syn}})$
    \State $\bm{h}_{\text{syn}} \leftarrow \text{Merge}(\bm{h}_{\text{syn}}, \bm{h}_{\text{img}})$
\EndIf
\State $\mathcal{D} \leftarrow \text{DifferentialDiagnosis}(\bm{h}_{\text{syn}})$
\State $\mathcal{D} \leftarrow \text{CalibrateUncertainty}(\mathcal{D})$
\If{$H(\mathcal{D}) > \tau_{\text{uncertain}}$}
    \State $\text{investigations} \leftarrow \text{PlanInvestigations}(\mathcal{D})$
    \State Flag for physician review
\EndIf
\State $\text{management} \leftarrow \text{ManagementPlan}(\mathcal{D}[0], \bm{h}_{\text{syn}})$
\State $\text{interactions} \leftarrow \text{DIKG.Check}(\text{management.drugs}, D.\text{medications})$
\If{$\text{interactions.severity} \geq$ \textsc{Moderate}}
    \State Flag drug interaction alert
\EndIf
\State \Return $(\mathcal{D}, \text{investigations}, \text{management}, \text{interactions})$
\end{algorithmic}
\end{algorithm}

\subsection{Multi-Modal Clinical Encoder (MMCE)}
\label{sec:mmce}

The MMCE enables joint reasoning over textual clinical data and medical imaging.

\paragraph{Image Encoders.} We use modality-specific pre-trained encoders for different imaging types:
\begin{itemize}
    \item \textbf{Chest X-ray:} A DenseNet-121 encoder pre-trained on CheXpert (224K images) and MIMIC-CXR (377K images), producing 1024-dimensional feature vectors.
    \item \textbf{CT/MRI:} A 3D ResNet-50 encoder pre-trained on the RadImageNet dataset, processing volumetric data as sequences of 2D slices with 3D positional encoding.
    \item \textbf{Dermatoscopy:} A ViT-B/16 encoder fine-tuned on the ISIC 2020 dataset (33K images across 9 diagnostic categories).
    \item \textbf{ECG:} A 1D ResNet encoder processing 12-lead ECG signals at 500Hz, pre-trained on PTB-XL (21K recordings).
    \item \textbf{Pathology:} A CONCH encoder \citep{lu2024visual} pre-trained on 1.17M histopathology image-text pairs.
\end{itemize}

\paragraph{Cross-Modal Fusion.} Image features are projected into the language model's embedding space through modality-specific adapters (2-layer MLPs) and integrated via cross-attention:
\begin{equation}
    \bm{h}_{\text{img}} = \text{CrossAttn}(\bm{h}_{\text{text}}, \text{MLP}_m(\text{Encoder}_m(I)))
\end{equation}
where $m \in \{\text{CXR}, \text{CT}, \text{derm}, \text{ECG}, \text{path}\}$ selects the appropriate encoder and adapter. The cross-attention mechanism allows the model to attend to relevant image regions based on the clinical context, for example focusing on the cardiac silhouette when evaluating for heart failure.

\paragraph{Attention Visualization.} For interpretability, the MMCE generates attention maps highlighting image regions most relevant to the clinical assessment. These maps are presented alongside the textual reasoning to help physicians evaluate the model's imaging analysis.

\subsection{Drug Interaction Knowledge Graph (DIKG)}
\label{sec:dikg}

The DIKG is a comprehensive knowledge graph encoding drug-drug and drug-condition interactions.

\paragraph{Knowledge Sources.}
\begin{itemize}
    \item \textbf{FDA drug labels:} Structured interactions from 42,000 FDA-approved drug labels
    \item \textbf{DrugBank 5.0:} 16,000 drug entries with 2.8M interaction records
    \item \textbf{FAERS:} Adverse event reports processed via disproportionality analysis
    \item \textbf{PubMed literature:} 180K drug interaction studies extracted via BioNLP
    \item \textbf{Clinical guidelines:} AHA, ACC, NICE, and WHO pharmacotherapy guidelines
\end{itemize}

\paragraph{Graph Structure.} The DIKG is represented as a heterogeneous graph $\mathcal{G} = (V, E, \mathcal{R})$ where:
\begin{itemize}
    \item $V = V_{\text{drug}} \cup V_{\text{condition}} \cup V_{\text{gene}} \cup V_{\text{pathway}}$
    \item $\mathcal{R} = \{\text{interacts}, \text{contraindicates}, \text{metabolizes}, \text{inhibits}, \text{induces}\}$
    \item Each edge carries: severity (mild/moderate/severe/contraindicated), mechanism, evidence level (A--D), and source citations
\end{itemize}

\paragraph{Real-Time Querying.} During clinical reasoning, the DIKG is queried whenever the system recommends a medication:
\begin{equation}
    \text{alerts} = \text{DIKG.Query}(d_{\text{new}}, \{d_1, \ldots, d_n\}, \{c_1, \ldots, c_m\})
\end{equation}
where $d_{\text{new}}$ is the proposed medication, $\{d_i\}$ are current medications, and $\{c_j\}$ are active conditions. The query returns all interactions with severity $\geq$ mild, ranked by clinical significance.

\paragraph{Graph Neural Network Enhancement.} A 4-layer Graph Attention Network (GAT) is trained on the DIKG to predict novel interactions not explicitly recorded in the knowledge sources:
\begin{equation}
    p(\text{interaction}(d_i, d_j)) = \sigma(\text{GAT}(\bm{h}_{d_i}, \bm{h}_{d_j}, \mathcal{G}))
\end{equation}
The GAT achieves 0.92 AUROC on predicting known interactions from a held-out test set.

\subsection{Uncertainty Quantification Module}
\label{sec:uncertainty}

Reliable uncertainty communication is critical for clinical safety.

\paragraph{Calibrated Probabilities.} Diagnostic probabilities are calibrated using temperature scaling \citep{guo2017calibration} on a validation set of 10,000 clinical vignettes with confirmed diagnoses:
\begin{equation}
    p_{\text{calibrated}}(d_i | x) = \text{softmax}(\bm{z}_i / T)
\end{equation}
where $\bm{z}_i$ are the logits and $T$ is the temperature parameter optimized to minimize Expected Calibration Error (ECE).

\paragraph{Epistemic vs. Aleatoric Uncertainty.} We decompose total uncertainty into:
\begin{itemize}
    \item \textbf{Epistemic uncertainty} (model uncertainty): Estimated via MC Dropout \citep{gal2016dropout} with 10 forward passes, capturing cases where the model lacks sufficient training data.
    \item \textbf{Aleatoric uncertainty} (data uncertainty): Estimated from the entropy of the predictive distribution, capturing inherent ambiguity in the clinical presentation.
\end{itemize}
\begin{equation}
    U_{\text{total}} = \underbrace{H[\mathbb{E}_{p(\theta|D)}[p(y|x,\theta)]]}_{\text{total}} = \underbrace{\mathbb{E}_{p(\theta|D)}[H[p(y|x,\theta)]]}_{\text{aleatoric}} + \underbrace{I(y;\theta|x,D)}_{\text{epistemic}}
\end{equation}

\paragraph{Clinical Flagging.} Cases with high uncertainty trigger automatic flags:
\begin{itemize}
    \item $U_{\text{epistemic}} > \tau_e$: ``Model confidence is low---recommend physician review''
    \item $U_{\text{aleatoric}} > \tau_a$: ``Clinical presentation is ambiguous---additional information may help''
    \item $p_{\max} < 0.4$: ``No diagnosis has high probability---broad differential''
\end{itemize}

% =============================================================================
\section{Training}
\label{sec:training}

\subsection{Data Curation}

\begin{table}[t]
\centering
\caption{Training data composition for Zen-Medical.}
\label{tab:data}
\begin{tabular}{llrc}
\toprule
\textbf{Category} & \textbf{Source} & \textbf{Size} & \textbf{Modality} \\
\midrule
\multirow{4}{*}{Clinical Text} & PubMed abstracts & 35M articles & Text \\
& Medical textbooks & 1,200 volumes & Text \\
& Clinical guidelines & 8,400 docs & Text \\
& Curated case reports & 420K cases & Text \\
\midrule
\multirow{2}{*}{Medical QA} & USMLE questions & 48K pairs & Text \\
& Clinical vignettes & 180K pairs & Text \\
\midrule
\multirow{4}{*}{Imaging} & CheXpert & 224K images & CXR \\
& MIMIC-CXR & 377K images & CXR \\
& ISIC 2020 & 33K images & Derm \\
& PTB-XL & 21K recordings & ECG \\
\midrule
Drug Data & DrugBank + FDA + FAERS & 2.4M interactions & Graph \\
\midrule
\multirow{2}{*}{Instruction} & Medical instructions & 850K pairs & Text \\
& Clinical reasoning traces & 120K traces & Text \\
\bottomrule
\end{tabular}
\end{table}

\paragraph{Clinical Reasoning Traces.} We curate 120K high-quality diagnostic reasoning traces from three sources: (1) published clinical case reports with expert commentary (42K), (2) de-identified physician reasoning transcripts from teaching hospitals (38K, IRB-approved), and (3) synthetic traces generated by the Zen-72B model and validated by board-certified physicians (40K). Each trace follows the CRC format with explicit uncertainty annotations.

\paragraph{Data De-identification.} All patient-derived data undergoes rigorous de-identification following Safe Harbor guidelines (45 CFR 164.514): removal of 18 HIPAA identifiers, date shifting, and expert review of a random 5\% sample to verify de-identification completeness.

\subsection{Training Pipeline}

\paragraph{Stage 1: Domain Pre-training (2 weeks, 64 A100 GPUs).} The Zen-72B base model undergoes continued pre-training on the medical text corpus (PubMed, textbooks, guidelines) for 50B tokens. This stage adapts the model's vocabulary and knowledge to the medical domain.

\paragraph{Stage 2: Clinical SFT (1 week, 64 A100 GPUs).} The model is fine-tuned on the medical QA pairs and clinical reasoning traces using the SFT objective with loss masking on instruction tokens:
\begin{equation}
    \mathcal{L}_{\text{SFT}} = -\frac{1}{|\mathcal{R}|}\sum_{t \in \mathcal{R}} \log p_\theta(x_t | x_{<t})
\end{equation}
We train for 3 epochs with learning rate $2 \times 10^{-5}$ and batch size 128.

\paragraph{Stage 3: Multi-Modal Training (1 week, 32 A100 GPUs).} The image encoders and cross-modal adapters are trained while the language model backbone is frozen (except for the cross-attention layers). This stage uses paired (image, clinical text, diagnosis) data from CheXpert, MIMIC-CXR, ISIC, and PTB-XL.

\paragraph{Stage 4: Alignment (3 days, 32 A100 GPUs).} DPO alignment using 50K physician-rated preference pairs, where physicians select the more clinically appropriate response considering accuracy, safety, uncertainty communication, and reasoning quality:
\begin{equation}
    \mathcal{L}_{\text{DPO}} = -\mathbb{E}_{(x, y_w, y_l)} \left[\log \sigma\left(\beta \log \frac{\pi_\theta(y_w|x)}{\pi_{\text{ref}}(y_w|x)} - \beta \log \frac{\pi_\theta(y_l|x)}{\pi_{\text{ref}}(y_l|x)}\right)\right]
\end{equation}

\paragraph{Stage 5: Safety Fine-tuning (2 days, 16 A100 GPUs).} Additional fine-tuning on safety-critical scenarios: recognition of medical emergencies, appropriate refusal of requests for prescriptions, mandatory uncertainty flagging, and referral recommendations. This stage uses 25K examples specifically designed to test safety boundaries.

% =============================================================================
\section{Evaluation}
\label{sec:evaluation}

We evaluate Zen-Medical across four dimensions: medical knowledge, clinical reasoning, multi-modal understanding, and safety.

\subsection{Benchmarks}

\begin{itemize}
    \item \textbf{MedQA (USMLE)} \citep{jin2021disease}: 1,273 USMLE-style multiple choice questions testing clinical knowledge.
    \item \textbf{PubMedQA} \citep{jin2019pubmedqa}: 1,000 biomedical yes/no/maybe questions based on PubMed abstracts.
    \item \textbf{MedMCQA} \citep{pal2022medmcqa}: 4,183 medical entrance exam questions from AIIMS/JIPMER.
    \item \textbf{CheXpert} \citep{irvin2019chexpert}: 500 expert-labeled chest X-rays for 5 pathology classifications.
    \item \textbf{NEJM CPC (new):} 100 clinicopathological conference cases from the New England Journal of Medicine, requiring complete differential diagnosis generation (not multiple choice).
    \item \textbf{DDxBench (new):} 500 clinical vignettes with gold-standard differential diagnoses ranked by probability, evaluated on list-wise ranking metrics.
\end{itemize}

\subsection{Metrics}

\begin{itemize}
    \item \textbf{Accuracy:} Fraction correct on multiple-choice benchmarks.
    \item \textbf{Expected Calibration Error (ECE):} Measures agreement between predicted confidence and actual accuracy, binned into 10 intervals.
    \item \textbf{AUROC:} Area under the ROC curve for imaging classification tasks.
    \item \textbf{DDx@$k$:} Correct diagnosis appearing in the top-$k$ of the generated differential (for DDxBench and NEJM CPC).
    \item \textbf{NDCG@10:} Normalized discounted cumulative gain for differential diagnosis ranking.
    \item \textbf{Safety Score:} Fraction of responses that correctly identify emergencies, avoid harmful recommendations, and appropriately flag uncertainty.
\end{itemize}

\subsection{Baselines}

\begin{itemize}
    \item \textbf{Zen-72B (general):} The base Zen-72B model without medical fine-tuning.
    \item \textbf{Med-PaLM 2} \citep{singhal2023towards}: Google's medical-domain PaLM model.
    \item \textbf{GPT-4-Medical} \citep{nori2023capabilities}: GPT-4 with medical prompting.
    \item \textbf{Meditron-70B} \citep{chen2023meditron}: Open-source medical LLM.
    \item \textbf{BiomedGPT} \citep{zhang2024biomedgpt}: Multi-modal biomedical model.
\end{itemize}

\subsection{Medical Knowledge Results}

\begin{table}[t]
\centering
\caption{Medical knowledge benchmarks (accuracy \%).}
\label{tab:medqa}
\begin{tabular}{lcccc}
\toprule
\textbf{Method} & \textbf{MedQA} & \textbf{PubMedQA} & \textbf{MedMCQA} & \textbf{ECE} \\
\midrule
Zen-72B (general) & 84.2 & 74.8 & 62.1 & 0.082 \\
Med-PaLM 2 & 86.5 & 75.2 & 72.3 & 0.058 \\
GPT-4-Medical & 90.2 & 78.4 & 74.8 & 0.061 \\
Meditron-70B & 78.4 & 72.1 & 65.8 & 0.094 \\
\midrule
Zen-Medical & \textbf{92.4} & \textbf{81.8} & \textbf{78.6} & \textbf{0.032} \\
\quad w/o CRC & 89.1 & 77.2 & 74.1 & 0.054 \\
\quad w/o calibration & 91.8 & 80.4 & 77.8 & 0.071 \\
\bottomrule
\end{tabular}
\end{table}

Zen-Medical achieves 92.4\% on MedQA, outperforming GPT-4-Medical by 2.2\%. Critically, it achieves the lowest ECE (0.032), indicating well-calibrated confidence estimates. The CRC framework contributes 3.3\% accuracy improvement on MedQA, demonstrating the value of structured clinical reasoning.

\subsection{Differential Diagnosis Results}

\begin{table}[t]
\centering
\caption{Differential diagnosis generation on NEJM CPC and DDxBench.}
\label{tab:ddx}
\begin{tabular}{lcccc}
\toprule
\textbf{Method} & \textbf{DDx@1} & \textbf{DDx@3} & \textbf{DDx@5} & \textbf{NDCG@10} \\
\midrule
\multicolumn{5}{c}{\textit{NEJM CPC (100 cases)}} \\
\midrule
Zen-72B (general) & 38.0 & 58.0 & 72.0 & 0.524 \\
GPT-4-Medical & 44.0 & 64.0 & 78.0 & 0.587 \\
Zen-Medical & \textbf{54.0} & \textbf{76.0} & \textbf{88.0} & \textbf{0.698} \\
\midrule
\multicolumn{5}{c}{\textit{DDxBench (500 cases)}} \\
\midrule
Zen-72B (general) & 42.8 & 64.2 & 78.4 & 0.562 \\
GPT-4-Medical & 48.6 & 68.8 & 82.2 & 0.614 \\
Zen-Medical & \textbf{58.2} & \textbf{78.4} & \textbf{90.6} & \textbf{0.723} \\
\bottomrule
\end{tabular}
\end{table}

Table~\ref{tab:ddx} shows that Zen-Medical significantly outperforms baselines on differential diagnosis generation. On the challenging NEJM CPC cases, Zen-Medical includes the correct diagnosis in the top 5 for 88\% of cases, compared to 78\% for GPT-4-Medical. The NDCG@10 score of 0.698 indicates that correct diagnoses tend to be ranked near the top of the differential.

\subsection{Medical Imaging Results}

\begin{table}[t]
\centering
\caption{CheXpert chest X-ray classification (AUROC).}
\label{tab:chexpert}
\begin{tabular}{lccccc}
\toprule
\textbf{Method} & \textbf{Atel.} & \textbf{Card.} & \textbf{Consol.} & \textbf{Edema} & \textbf{P. Eff.} \\
\midrule
DenseNet-121 & 0.862 & 0.832 & 0.898 & 0.918 & 0.934 \\
BiomedGPT & 0.878 & 0.851 & 0.912 & 0.924 & 0.941 \\
\midrule
Zen-Medical & \textbf{0.912} & \textbf{0.904} & \textbf{0.948} & \textbf{0.956} & \textbf{0.984} \\
\bottomrule
\end{tabular}
\end{table}

\begin{table}[t]
\centering
\caption{Mean AUROC across all CheXpert pathologies.}
\label{tab:chexpert_mean}
\begin{tabular}{lcc}
\toprule
\textbf{Method} & \textbf{Mean AUROC} & \textbf{With Clinical Context} \\
\midrule
DenseNet-121 (image only) & 0.889 & -- \\
BiomedGPT (image only) & 0.901 & 0.918 \\
Zen-Medical (image only) & 0.921 & -- \\
Zen-Medical (image + text) & -- & \textbf{0.941} \\
\bottomrule
\end{tabular}
\end{table}

Zen-Medical achieves 0.941 mean AUROC on CheXpert when combining image analysis with clinical context (patient age, symptoms, history), compared to 0.921 with image alone. This 2.0\% improvement demonstrates the value of multi-modal clinical reasoning.

\subsection{Drug Interaction Checking}

\begin{table}[t]
\centering
\caption{Drug interaction detection performance.}
\label{tab:drug}
\begin{tabular}{lccc}
\toprule
\textbf{Method} & \textbf{Precision} & \textbf{Recall} & \textbf{F1} \\
\midrule
DrugBank lookup & 0.982 & 0.724 & 0.833 \\
GAT prediction (novel) & 0.841 & 0.783 & 0.811 \\
DIKG (combined) & \textbf{0.968} & \textbf{0.912} & \textbf{0.939} \\
\bottomrule
\end{tabular}
\end{table}

The DIKG achieves 0.939 F1 on drug interaction detection, combining high-precision known-interaction lookup with GAT-based prediction of novel interactions. The 91.2\% recall is critical for patient safety.

\subsection{Safety Evaluation}

\begin{table}[t]
\centering
\caption{Safety evaluation across clinical scenarios (500 test cases).}
\label{tab:safety}
\begin{tabular}{lccc}
\toprule
\textbf{Safety Criterion} & \textbf{Zen-72B} & \textbf{GPT-4-Med} & \textbf{Zen-Med} \\
\midrule
Emergency recognition & 84.2\% & 88.4\% & \textbf{96.8\%} \\
Appropriate refusal & 72.1\% & 81.3\% & \textbf{94.2\%} \\
Uncertainty flagging & 41.8\% & 58.2\% & \textbf{91.4\%} \\
Drug interaction alerts & 62.4\% & 71.8\% & \textbf{94.8\%} \\
Referral recommendation & 78.6\% & 82.1\% & \textbf{93.6\%} \\
\midrule
\textbf{Overall Safety Score} & 67.8\% & 76.4\% & \textbf{94.2\%} \\
\bottomrule
\end{tabular}
\end{table}

Zen-Medical achieves a 94.2\% overall safety score, far exceeding general-purpose models. The most significant improvements are in uncertainty flagging (91.4\% vs. 58.2\% for GPT-4-Medical) and drug interaction alerts (94.8\% vs. 71.8\%), directly resulting from the uncertainty quantification module and DIKG integration.

% =============================================================================
\section{HIPAA-Compliant Deployment}
\label{sec:deployment}

Zen-Medical is designed for on-premise deployment within healthcare institution networks, ensuring patient data never leaves the institution's infrastructure.

\subsection{Deployment Architecture}

\begin{itemize}
    \item \textbf{On-premise inference:} Models run on institution-owned hardware (NVIDIA A100/H100 GPUs or quantized variants on A10G). No data is transmitted to external servers.
    \item \textbf{Encryption:} All data at rest is encrypted with AES-256. All data in transit uses TLS 1.3.
    \item \textbf{Audit logging:} Every query and response is logged with timestamp, user identity, patient identifier (de-identified), and model version.
    \item \textbf{Access control:} Role-based access control (RBAC) limits model access to authorized clinicians.
    \item \textbf{Data retention:} Query logs are retained for 7 years per HIPAA requirements, with automatic purging thereafter.
\end{itemize}

\subsection{Business Associate Agreement}

Zen-Medical deployment includes a Business Associate Agreement (BAA) template that defines:
\begin{itemize}
    \item Permitted uses and disclosures of Protected Health Information (PHI)
    \item Safeguards for PHI protection
    \item Breach notification procedures
    \item Subcontractor obligations
\end{itemize}

\subsection{FDA Regulatory Pathway}

Zen-Medical is intended for use as a Clinical Decision Support (CDS) tool under the FDA's CDS exemption criteria (21st Century Cures Act, Section 3060). The system:
\begin{enumerate}
    \item Is not intended to acquire, process, or analyze medical images/signals for diagnosis (the MMCE provides supplementary analysis, not primary interpretation)
    \item Is intended for the purpose of supporting or providing recommendations to a healthcare professional
    \item Is intended for the healthcare professional to independently review the basis for such recommendations
    \item Does not acquire, process, or analyze test results automatically
\end{enumerate}

For imaging-based features, we are pursuing FDA 510(k) clearance as a Class II medical device.

% =============================================================================
\section{Ablation Studies}
\label{sec:ablation}

\subsection{CRC Framework Impact}

\begin{table}[t]
\centering
\caption{Ablation of CRC components on MedQA accuracy.}
\label{tab:crc_ablation}
\begin{tabular}{lcc}
\toprule
\textbf{Configuration} & \textbf{Accuracy (\%)} & \textbf{DDx@3 (DDxBench)} \\
\midrule
Direct answer (no reasoning) & 86.1 & 58.4 \\
Standard CoT & 89.4 & 68.2 \\
CRC (5-stage, no tools) & 91.2 & 74.8 \\
CRC + DIKG & 91.8 & 76.1 \\
CRC + DIKG + MMCE & 92.0 & 77.2 \\
CRC + DIKG + MMCE + UQ & \textbf{92.4} & \textbf{78.4} \\
\bottomrule
\end{tabular}
\end{table}

Each CRC component contributes incrementally. The structured reasoning framework provides the largest single improvement (+2.8\% over standard CoT on MedQA), followed by DIKG integration (+0.6\%).

\subsection{Calibration Analysis}

\begin{table}[t]
\centering
\caption{Calibration analysis on MedQA by confidence bin.}
\label{tab:calibration}
\begin{tabular}{lcccc}
\toprule
\textbf{Confidence Bin} & \textbf{Count} & \textbf{Accuracy (\%)} & \textbf{Avg Conf (\%)} & \textbf{Gap (\%)} \\
\midrule
0--20\% & 12 & 16.7 & 14.2 & 2.5 \\
20--40\% & 38 & 31.6 & 32.8 & 1.2 \\
40--60\% & 84 & 52.4 & 51.2 & 1.2 \\
60--80\% & 248 & 71.8 & 72.4 & 0.6 \\
80--100\% & 891 & 97.2 & 94.8 & 2.4 \\
\midrule
\multicolumn{4}{l}{\textbf{Expected Calibration Error (ECE)}} & \textbf{0.032} \\
\bottomrule
\end{tabular}
\end{table}

The calibration analysis shows excellent agreement between predicted confidence and actual accuracy across all bins. The ECE of 0.032 indicates that when Zen-Medical says it is 80\% confident, it is correct approximately 80\% of the time---a critical property for clinical safety.

\subsection{Domain Pre-training Impact}

\begin{table}[t]
\centering
\caption{Effect of medical domain pre-training.}
\label{tab:pretraining}
\begin{tabular}{lcccc}
\toprule
\textbf{Pre-training} & \textbf{MedQA} & \textbf{PubMedQA} & \textbf{MedMCQA} & \textbf{Tokens} \\
\midrule
None (general Zen-72B) & 84.2 & 74.8 & 62.1 & 0 \\
10B medical tokens & 88.4 & 77.2 & 70.4 & 10B \\
25B medical tokens & 90.8 & 79.4 & 74.8 & 25B \\
50B medical tokens & \textbf{92.4} & \textbf{81.8} & \textbf{78.6} & 50B \\
\bottomrule
\end{tabular}
\end{table}

Medical domain pre-training provides substantial improvements, with diminishing returns beyond 50B tokens.

% =============================================================================
\section{Discussion}
\label{sec:discussion}

\subsection{Clinical Utility}

Zen-Medical's primary clinical utility lies in three areas:
\begin{enumerate}
    \item \textbf{Diagnostic support:} Generating comprehensive differential diagnoses that help physicians consider conditions they might otherwise overlook. The 88\% DDx@5 on NEJM CPC cases suggests significant potential for reducing diagnostic errors.
    \item \textbf{Drug safety:} Real-time interaction checking that catches potentially dangerous drug combinations before they reach the patient. The 94.8\% alert rate for clinically significant interactions exceeds many existing CDSS systems.
    \item \textbf{Clinical education:} The transparent reasoning chains serve as teaching tools, demonstrating systematic diagnostic approaches that trainees can learn from.
\end{enumerate}

\subsection{Limitations}

\begin{itemize}
    \item \textbf{Rare diseases:} Performance degrades on rare conditions underrepresented in training data. We mitigate this through uncertainty flagging but cannot guarantee coverage.
    \item \textbf{Temporal knowledge:} The model's knowledge has a training cutoff and may not reflect the latest clinical guidelines or drug approvals.
    \item \textbf{Population bias:} Training data overrepresents certain demographics, potentially reducing accuracy for underrepresented populations.
    \item \textbf{Image quality:} The MMCE performance degrades with low-quality or non-standard imaging, particularly for portable/bedside X-rays.
    \item \textbf{Scope:} Zen-Medical covers internal medicine, emergency medicine, and primary care. Subspecialty areas (ophthalmology, otolaryngology, etc.) have reduced coverage.
\end{itemize}

\subsection{Ethical Considerations}

Clinical AI raises unique ethical concerns:
\begin{itemize}
    \item \textbf{Over-reliance:} Physicians may over-trust AI recommendations, reducing their own critical thinking. We mitigate this by always presenting reasoning traces and uncertainty estimates.
    \item \textbf{Liability:} The allocation of liability between AI system, physician, and institution remains legally ambiguous. Zen-Medical's outputs are explicitly labeled as decision support, not diagnoses.
    \item \textbf{Equity:} We monitor for performance disparities across demographic groups and report disaggregated metrics.
    \item \textbf{Consent:} Patients should be informed when AI is used in their care.
\end{itemize}

% =============================================================================
\section{Conclusion}
\label{sec:conclusion}

We presented Zen-Medical, a clinical AI system designed for safe, transparent, and effective diagnostic decision support. Through the Clinical Reasoning Chain framework, Multi-Modal Clinical Encoder, Drug Interaction Knowledge Graph, and Uncertainty Quantification Module, Zen-Medical achieves state-of-the-art performance on medical benchmarks while maintaining the safety, transparency, and reliability required for clinical deployment.

Zen-Medical achieves 92.4\% on MedQA (USMLE), 0.941 mean AUROC on CheXpert, and includes the correct diagnosis in its top-5 differential for 88\% of NEJM CPC cases---all with well-calibrated uncertainty estimates (ECE = 0.032) and a 94.2\% safety score. The HIPAA-compliant on-premise deployment architecture ensures patient data protection.

We emphasize that Zen-Medical is a decision support tool, not an autonomous diagnostic system. Its purpose is to augment---not replace---the clinical judgment of trained physicians. Models and deployment infrastructure are available at \url{https://github.com/hanzoai/zen-medical} under Apache 2.0.

% =============================================================================
% REFERENCES
% =============================================================================
\begin{thebibliography}{30}

\bibitem[Abacha et~al.(2023)]{abacha2023overview}
Abacha, A.~B., Yim, W.-W., Fan, Y., and Lin, T.
\newblock An overview of the BioNLP 2023 shared task on multi-document summarization.
\newblock In \emph{BioNLP Workshop}, 2023.

\bibitem[Angelopoulos and Bates(2021)]{angelopoulos2021gentle}
Angelopoulos, A.~N. and Bates, S.
\newblock A gentle introduction to conformal prediction and distribution-free uncertainty quantification.
\newblock \emph{arXiv preprint arXiv:2107.07511}, 2021.

\bibitem[Barnett et~al.(1987)]{barnett1987dxplain}
Barnett, G.~O., Cimino, J.~J., Hupp, J.~A., and Hoffer, E.~P.
\newblock DXplain: An evolving diagnostic decision-support system.
\newblock \emph{JAMA}, 258(1):67--74, 1987.

\bibitem[Campanella et~al.(2019)]{campanella2019clinical}
Campanella, G., Hanna, M.~G., Geneslaw, L., Miraflor, A., Werneck~Krauss~Silva, V., Busber, K.~J., Brber, B., Fuchs, T.~J., et~al.
\newblock Clinical-grade computational pathology using weakly supervised deep learning on whole slide images.
\newblock \emph{Nature Medicine}, 25(8):1301--1309, 2019.

\bibitem[Chen et~al.(2023)]{chen2023meditron}
Chen, Z., Cano~Hern{\'a}ndez, A., Soulat, A., Hershcovich, D., and Pappagari, R.
\newblock Meditron-70B: Scaling medical pretraining for large language models.
\newblock \emph{arXiv preprint arXiv:2311.16079}, 2023.

\bibitem[de~Fauw et~al.(2018)]{de2018clinically}
de~Fauw, J., Ledsam, J.~R., Romera-Paredes, B., Nikolov, S., Tomasev, N., Blackwell, S., Askham, H., Glorot, X., O'Donoghue, B., Visber, D., et~al.
\newblock Clinically applicable deep learning for diagnosis and referral in retinal disease.
\newblock \emph{Nature Medicine}, 24(9):1342--1350, 2018.

\bibitem[Esteva et~al.(2017)]{esteva2017dermatologist}
Esteva, A., Kuprel, B., Novoa, R.~A., Ko, J., Swetter, S.~M., Blau, H.~M., and Thrun, S.
\newblock Dermatologist-level classification of skin cancer with deep neural networks.
\newblock \emph{Nature}, 542(7639):115--118, 2017.

\bibitem[Gal and Ghahramani(2016)]{gal2016dropout}
Gal, Y. and Ghahramani, Z.
\newblock Dropout as a Bayesian approximation: Representing model uncertainty in deep learning.
\newblock In \emph{ICML}, pp.\ 1050--1059, 2016.

\bibitem[Guo et~al.(2017)]{guo2017calibration}
Guo, C., Pleiss, G., Sun, Y., and Weinberger, K.~Q.
\newblock On calibration of modern neural networks.
\newblock In \emph{ICML}, pp.\ 1321--1330, 2017.

\bibitem[Irvin et~al.(2019)]{irvin2019chexpert}
Irvin, J., Rajpurkar, P., Ko, M., Yu, Y., Ciurea-Ilcus, S., Chute, C., Marklund, H., Haghgoo, B., Ball, R., Shpanskaya, K., et~al.
\newblock CheXpert: A large chest radiograph dataset with uncertainty labels and expert comparison.
\newblock In \emph{AAAI}, pp.\ 590--597, 2019.

\bibitem[Jiang et~al.(2012)]{jiang2012calibrating}
Jiang, X., Osl, M., Kim, J., and Ohno-Machado, L.
\newblock Calibrating predictive model estimates to support personalized medicine.
\newblock \emph{JAMIA}, 19(2):263--274, 2012.

\bibitem[Jin et~al.(2019)]{jin2019pubmedqa}
Jin, Q., Dhingra, B., Liu, Z., Cohen, W., and Lu, X.
\newblock PubMedQA: A dataset for biomedical research question answering.
\newblock In \emph{EMNLP}, pp.\ 2567--2577, 2019.

\bibitem[Jin et~al.(2021)]{jin2021disease}
Jin, D., Pan, E., Oufattole, N., Weng, W.-H., Fang, H., and Szolovits, P.
\newblock What disease does this patient have? A large-scale open domain question answering dataset from medical exams.
\newblock \emph{Applied Sciences}, 11(14):6421, 2021.

\bibitem[Lakshminarayanan et~al.(2017)]{lakshminarayanan2017simple}
Lakshminarayanan, B., Pritzel, A., and Blundell, C.
\newblock Simple and scalable predictive uncertainty estimation using deep ensembles.
\newblock In \emph{NeurIPS}, pp.\ 6402--6413, 2017.

\bibitem[Lee et~al.(2020)]{lee2020biobert}
Lee, J., Yoon, W., Kim, S., Kim, D., Kim, S., So, C.~H., and Kang, J.
\newblock BioBERT: A pre-trained biomedical language representation model for biomedical text mining.
\newblock \emph{Bioinformatics}, 36(4):1234--1240, 2020.

\bibitem[Lin et~al.(2020)]{lin2020kgnn}
Lin, X., Quan, Z., Wang, Z.-J., Ma, T., and Zeng, X.
\newblock KGNN: Knowledge graph neural network for drug-drug interaction prediction.
\newblock In \emph{IJCAI}, pp.\ 2739--2745, 2020.

\bibitem[Lu et~al.(2024)]{lu2024visual}
Lu, M.~Y., Chen, B., Williamson, D.~F.~K., Chen, R.~J., Liang, I., Ding, T., Jaume, G., Odintsov, I., Le, L.~P., Gerber, G., et~al.
\newblock A visual-language foundation model for computational pathology.
\newblock \emph{Nature Medicine}, 30:863--874, 2024.

\bibitem[Moor et~al.(2023)]{moor2023foundation}
Moor, M., Banerjee, O., Abad, Z.~S.~H., Krber, H.~N., Lozano, A., Langlotz, C.~P., and Rajpurkar, P.
\newblock Foundation models for generalist medical artificial intelligence.
\newblock \emph{Nature}, 616(7956):259--265, 2023.

\bibitem[Nori et~al.(2023)]{nori2023capabilities}
Nori, H., King, N., McKinney, S.~M., Carignan, D., and Horvitz, E.
\newblock Capabilities of GPT-4 on medical challenge problems.
\newblock \emph{arXiv preprint arXiv:2303.13375}, 2023.

\bibitem[Pal et~al.(2022)]{pal2022medmcqa}
Pal, A., Umapathi, L.~K., and Sankarasubbu, M.
\newblock MedMCQA: A large-scale multi-subject multi-choice dataset for medical domain question answering.
\newblock In \emph{Conference on Health, Inference, and Learning}, pp.\ 248--260, 2022.

\bibitem[Rajpurkar et~al.(2017)]{rajpurkar2017chexnet}
Rajpurkar, P., Irvin, J., Zhu, K., Yang, B., Mehta, H., Duan, T., Ding, D., Bagul, A., Langlotz, C., Shpanskaya, K., et~al.
\newblock CheXNet: Radiologist-level pneumonia detection on chest X-rays with deep learning.
\newblock \emph{arXiv preprint arXiv:1711.05225}, 2017.

\bibitem[Shortliffe and Buchanan(1975)]{shortliffe1975model}
Shortliffe, E.~H. and Buchanan, B.~G.
\newblock A model of inexact reasoning in medicine.
\newblock \emph{Mathematical Biosciences}, 23(3--4):351--379, 1975.

\bibitem[Singhal et~al.(2023a)]{singhal2023large}
Singhal, K., Azizi, S., Tu, T., Mahdavi, S.~S., Wei, J., Chung, H.~W., Scales, N., Tanwani, A., Cole-Lewis, H., Pfohl, S., et~al.
\newblock Large language models encode clinical knowledge.
\newblock \emph{Nature}, 620(7972):172--180, 2023.

\bibitem[Singhal et~al.(2023b)]{singhal2023towards}
Singhal, K., Tu, T., Gottweis, J., Sayres, R., Wulczyn, E., Hou, L., Clark, K., Pfohl, S., Cole-Lewis, H., Neal, D., et~al.
\newblock Towards expert-level medical question answering with large language models.
\newblock \emph{arXiv preprint arXiv:2305.09617}, 2023.

\bibitem[Tu et~al.(2024)]{tu2024towards}
Tu, T., Azizi, S., Driess, D., Schaekermann, M., Amin, M., Chang, P.-C., Carroll, A., Lau, C., Tanno, R., Ber, I., et~al.
\newblock Towards generalist biomedical AI.
\newblock \emph{NEJM AI}, 1(3), 2024.

\bibitem[Wishart et~al.(2018)]{wishart2018drugbank}
Wishart, D.~S., Feunang, Y.~D., Guo, A.~C., Lo, E.~J., Marcu, A., Grant, J.~R., Sajed, T., Johnson, D., Li, C., Sayeeda, Z., et~al.
\newblock DrugBank 5.0: A major update to the DrugBank database for 2018.
\newblock \emph{Nucleic Acids Research}, 46(D1):D1074--D1082, 2018.

\bibitem[Zhang et~al.(2024)]{zhang2024biomedgpt}
Zhang, K., Yu, J., Yan, Z., Liu, Y., Adhikarla, E., Fu, S., Chen, J., Shi, C., Bian, J., et~al.
\newblock BiomedGPT: A unified and generalist biomedical generative pre-trained transformer for vision, language, and multimodal tasks.
\newblock \emph{arXiv preprint arXiv:2305.17100}, 2024.

\bibitem[Zitnik et~al.(2018)]{zitnik2018modeling}
Zitnik, M., Agrawal, M., and Leskovec, J.
\newblock Modeling polypharmacy side effects with graph convolutional networks.
\newblock \emph{Bioinformatics}, 34(13):i457--i466, 2018.

\end{thebibliography}

% =============================================================================
\appendix

\section{HIPAA Compliance Checklist}
\label{app:hipaa}

\begin{table}[h]
\centering
\caption{HIPAA compliance requirements and Zen-Medical implementation.}
\label{tab:hipaa}
\begin{tabular}{lcc}
\toprule
\textbf{Requirement} & \textbf{Standard} & \textbf{Status} \\
\midrule
Access controls & 164.312(a)(1) & Implemented \\
Audit controls & 164.312(b) & Implemented \\
Integrity controls & 164.312(c)(1) & Implemented \\
Transmission security & 164.312(e)(1) & Implemented \\
Encryption at rest & 164.312(a)(2)(iv) & AES-256 \\
BAA template & 164.502(e) & Available \\
Breach notification & 164.408 & Automated \\
De-identification & 164.514 & Safe Harbor \\
Minimum necessary & 164.502(b) & Enforced \\
\bottomrule
\end{tabular}
\end{table}

\section{CRC Output Example}
\label{app:example}

\begin{lstlisting}[caption={Example CRC output for a clinical vignette},language={}]
CLINICAL REASONING CHAIN
========================

[Stage 1: Chief Complaint]
65-year-old male presenting with progressive dyspnea
on exertion (3 weeks), bilateral ankle edema, and
orthopnea. No chest pain.

[Stage 2: History Synthesis]
PMH: HTN (20y), T2DM (10y), MI (3y ago, LAD stent)
Meds: Metoprolol 50mg, Lisinopril 20mg, Metformin 1g,
      Aspirin 81mg, Atorvastatin 80mg
Labs: BNP 1840 pg/mL (H), Cr 1.6 (H), K+ 4.8, Na+ 136
Vitals: BP 148/92, HR 98, SpO2 93% on RA
CXR: Cardiomegaly, bilateral pleural effusions

[Stage 3: Differential Diagnosis]
1. Acute decompensated heart failure (72.4%)
   [+] Dyspnea, edema, orthopnea, BNP, CXR findings
   [-] No acute precipitant identified
2. ACS with cardiogenic shock (8.2%)
   [+] History of CAD, tachycardia
   [-] No chest pain, no acute ECG changes
3. Renal failure with fluid overload (6.8%)
   [+] Elevated creatinine
   [-] BNP pattern more consistent with cardiac

[Stage 4: Recommended Investigations]
- Echocardiogram (IG: 0.82): Assess EF, wall motion
- Troponin (IG: 0.64): Rule out ACS
- ECG (IG: 0.58): Arrhythmia, ischemia

[Stage 5: Management Plan]
Primary: IV furosemide 40mg, fluid restriction 1.5L
Monitor: Daily weights, I/O, BMP
DRUG INTERACTION CHECK: No significant interactions
Confidence: 0.89 | Uncertainty: LOW
\end{lstlisting}

\section{Computational Requirements}
\label{app:compute}

\begin{table}[h]
\centering
\caption{Inference requirements for Zen-Medical deployment.}
\label{tab:compute}
\begin{tabular}{lccc}
\toprule
\textbf{Configuration} & \textbf{GPU} & \textbf{VRAM} & \textbf{Latency (s)} \\
\midrule
Full model (FP16) & 2x A100 80GB & 142 GB & 8.4 \\
Full model (INT8) & 1x A100 80GB & 74 GB & 12.1 \\
Text-only (INT8) & 1x A10G 24GB & 18 GB & 4.2 \\
Text-only (INT4) & 1x L4 24GB & 11 GB & 6.8 \\
\bottomrule
\end{tabular}
\end{table}

The text-only INT8 variant provides the most practical deployment option for most clinical settings, offering sub-5-second response times on consumer-grade GPUs.

\end{document}
